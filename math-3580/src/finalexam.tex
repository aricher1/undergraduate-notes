\documentclass[12pt]{article}
\usepackage[margin=1in]{geometry}
\usepackage{amsmath, amssymb, amsthm} 
\usepackage{graphicx} 
\usepackage{booktabs} 
\usepackage{hyperref}
\usepackage{multicol}
\usepackage{enumitem}
\usepackage{fancyhdr} 
\usepackage{xcolor} 

% Custom styles
\theoremstyle{definition}
\newtheorem*{theorem*}{Theorem}
\newtheorem*{corollary*}{Corollary}
\newtheorem{definition}{Definition}
[section]
\newenvironment{solution*}{\par\noindent\textbf{Solution.}\ }{\hfill$\square$\par}
\pagestyle{fancy}
\fancyhf{}

\lhead{MATH-3580 Review}
\rhead{Aidan Richer}
\cfoot{\thepage}

% Document info
\title{\textbf{MATH-3580 Final Exam Review}}
\author{Aidan Richer}
\date{November 2025}

% Start of document
\begin{document}

\maketitle
\noindent \textbf{The Final Exam contains eight questions covering topics in Chapter 3 (Parts 1 - 3, 5, and 6) and Chapter 4 (Parts 1 - 3, up to Example 5) in the lecture outlines.}
\tableofcontents
\bigskip

\section{Definitions}
\subsection{Metric Space}
In mathematics, space = set + structure(s).

\begin{definition}
Let \(X\) be a set. A function \(d: X \times X \rightarrow [0, \infty)\) is called a \textbf{metric} (or distance) on \(X\) if
\begin{enumerate}
    \item \(\forall x, y \in X, d(x,y) = 0 \iff x = y;\)
    \item \(\forall x, y \in X, d(x,y)=d(y,x);\)
    \item \(\forall x, y, z \in X, d(x,z) \leq d(x,y) + d(y,z).\) (Triangle inequality)
\end{enumerate}
In this case, \((X,d)\) is called a \textbf{metric space}.
\end{definition}

\subsection{Open Ball and Bounded Set}
\begin{definition}
    Let \((X,d)\) be a metric space, \(x \in X\), and \(r > 0\).
\begin{enumerate}
    \item Define \(B(x,r) = \{y \in X : d(x,y) < r \}\), called the \textbf{open ball} centered at x with radius r, or the \textbf{r-neighborhood of x}.
    \item A general \textbf{neighborhood of x} is a subset \(U\) of \(X\) such that \(B(x,r) \subseteq U\) for some \(r > 0\).
    \item A subset \(E \subseteq X\) of a metric space \((X,d)\) is \textbf{bounded} if there exists some \(x_0 \in X\) and \(M > 0\) such that \(d(x,x_0) \leq M\) for all \(x \in E\).
\end{enumerate}
\end{definition}

\subsection{Interior Points and Interior}
\begin{definition}
    Let \(E \subseteq X\). The \textbf{closure} of \(E\) is the set \(\overline{E} = E \cup E'\).
    \begin{enumerate}
        \item An element \(x\) of \(X\) is called an \textbf{interior point} of \(E\) if \(\exists r > 0, B(x,r) \subseteq E\).
        \item The \textbf{interior} of \(E\) is the set \(E^{\circ}\) of all interior points of E.
    \end{enumerate}
    By definition, we have \(E^{\circ} \subseteq E \subseteq \overline{E}\).
\end{definition}

\subsection{Open Set}
\begin{definition}
    A subset \(G \subseteq X\) is called \textbf{open} if \(\forall x \in G, \exists r>0, B(x,r) \subseteq G\). Note, \(\emptyset\) and \(X\) are open in \(X\). That is, a set is open if every point in it has an open ball around it that’s still entirely inside the set.
\end{definition}

\subsection{Limit Points and Derived Set}
\begin{definition}
    Let \(E \subseteq X\). 
    \begin{enumerate}
        \item \(x\) is called a \textbf{limit point} of \(E\) (or cluster point, or accumulation point) if \[\forall r>0, B(x,r) \cap E \text{ contains some } y \not=x\]
        Equivalently, 
        \[(B(x,r)-\{x\})\cap E \not= \emptyset\]
        \item We let \(E' = \) the set of all limit points of \(E\), called the \textbf{derived set} of \(E\).
    \end{enumerate}
\end{definition}

\subsection{Closed Set}
\begin{definition}
    A subset \(E \subseteq X\) is called \textbf{closed} if \(E' \subseteq E\), where \(E'\) is the derived set of E (the set of all limit points of E). Note, \(\emptyset\) and \(X\) are closed in \(X\). That is, a subset \(E \subseteq X\) is closed if every limit point of \(E\) belongs to \(E\).
    \begin{enumerate}
        \item \(E\) is open \(\iff X-E\) is closed;
        \item \(E\) is closed \(\iff X-E\) is open.
    \end{enumerate}
\end{definition}

\subsection{Closure}
\begin{definition}
    Let \(E \subseteq X.\) The \textbf{closure} of \(E\) is the set \(\overline{E} = E \cup E'\). 
\end{definition}

\subsection{Boundary Points and Boundary}
\begin{definition}
    Let \(X\) be a metric space, \(E \subseteq X\) and \(x \in X\).
    \begin{enumerate}
        \item \(x\) is called a \textbf{boundary point} of \(E\) if \(\forall r > 0, B(x,r) \cap E \not = \emptyset\) and \(B(x,r) \cap (X-E) \not = \emptyset\).
        \item We use \(\partial E\) to denote the set of all boundary points of \(E\), called the \textbf{boundary} of \(E\).
    \end{enumerate}
\end{definition}

\subsection{Open Cover}
\begin{definition}
    Let \(X\) be a metric space and let \(K \subseteq X\). A family \(\{G_{\alpha}\}\) of open sets in \(X\) is called an \textbf{open cover} of \(K\) if \(K \subseteq \bigcup_{\alpha}G_{\alpha}\). That is, every point of \(K\) lies in at least one of the open sets \(G_{\alpha}\).
\end{definition}

\subsection{Compact Set}
\begin{definition}
    The set \(K\) is called \textbf{compact} if every open cover of \(K\) has a finite \textbf{subcover} of \(K\). That is, if \(K \subseteq \bigcup_{\alpha}G_{\alpha}\), then \(\exists \alpha_1,\dots,\alpha_n\) such that \(K \subseteq \bigcup^n_{i=1}G_{\alpha}\).
\end{definition}

\subsection{Convergent Sequence}
\begin{definition}
    Let \((X,d)\) be a metric space and let \(\{x_n\}\) be a sequence in \(X\). We say that \(\{x_n\}\) is \textbf{convergent} if
    \[\exists x \in X \text{ such that } \underline{\forall \epsilon > 0, \exists N = N(\epsilon) \in \mathbb{N}, \forall n \geq N, d(x_n, x) < \epsilon}.\]
    In this case, we say that \(\{x_n\}\) converges to \(x\), and write \(x_n \rightarrow x\).
\end{definition}

\subsection{Divergent Sequence}
\begin{definition}
    We say that \(\{x_n\}\) is \textbf{divergent} if \(\{x_n\}\) is not convergent. That is, \(\forall x \in X, \{x_n\}\) does not converge to \(x\). The \(\epsilon\)-\(N\) description of divergence of \(\{x_n\}\) is
    \[\underline{\forall x \in X, \exists \epsilon_0 > 0, \forall N, \exists n \geq N, d(x_n,x) \geq \epsilon_0}.\]
\end{definition}

\subsection{Bounded Sequence}
\begin{definition}
    A sequence \(\{x_n\}\) in a metric space \((X,d)\) is \textbf{bounded} if there exists a point \(x_0 \in X\) and a number \(M > 0\) such that 
    \[d(x_n,x_0)\leq M \text{ for all } n \in \mathbb{N}\]
    In other words, all terms of the sequence lie within some fixed distance \(M\) of a single point \(x_0\). In simplest terms, a sequence is bounded if all its terms lie inside some ball of finite radius.
\end{definition}

\subsection{The \(\epsilon\)-\(N\) description of \(x_n \rightarrow x\)}
\begin{definition}
    The \(\epsilon\)-\(N\) description of \(x_n \rightarrow x\) is 
    \[\underline{\forall \epsilon > 0, \exists N=N(\epsilon) \in \mathbb{N}, \forall n \geq N, d(x_n,x) < \epsilon}.\]
\end{definition}

\subsection{The \(\epsilon\)-\(N\) description of \(x_n \not\rightarrow x\) }
\begin{definition}
    The \(\epsilon\)-\(N\) description of \(x_n \not \rightarrow x\) is
    \[\underline{\exists \epsilon_0 > 0, \forall N, \exists n \in N, d(x_n,x) \geq \epsilon_0}.\]
\end{definition}

\subsection{The \(\epsilon\)-\(N\) definition of a Cauchy sequence and its negation}
\begin{definition}
    Let \(\{x_n\}\) be a sequence in a metric space \((X,d). \ \{x_n\}\) is called \underline{Cauchy} if
    \[\underline{\forall \epsilon > 0, \exists N, \forall m, n \geq N, d(x_m,x_n)<\epsilon}\]
    So we can say it's negation: \(\{x_n\}\) is not Cauchy if and only if
    \[\underline{\exists \epsilon_0 > 0, \forall N, \exists m , n \geq N, d(x_m, x_n) \geq \epsilon_0}\]
\end{definition}

\subsection{Subsequence}
\begin{definition}
    Let \(\{x_n\}\) be a sequence. If \[\{n_k\}_{k\in\mathbb{N}} \text{ is a sequence in } \mathbb{N} \text{ such that } n_1 < n_2 < \dots\]
    then \(\{x_{n_k}\}^\infty_{k=1}\) is called a \underline{subsequence} of \(\{x_n\}\).
\end{definition}

\subsection{Convergence, Divergence, Absolute Convergence of a Series}
\begin{definition}
    Let \(\{a_n\}\) be a sequence in \(\mathbb{R}\). For each \(n \in \mathbb{N}\), let
    \[s_n=a_1+\dots+a_n=\sum^n_{k=1}a_k\]
    called the \underline{\(n^{th}\) partial sum} of the series \(\displaystyle \sum^\infty_{k=1}a_k\).
    \newline \textbf{Convergence:} If \(s_n \to s \in \mathbb{R}\), then we say that the series \(\displaystyle \sum^\infty_{k=1}a_k\) is \underline{convergent}, and we write \[\sum^\infty_{k=1}a_k=s \text{ (called the sum of } \sum^\infty_{k=1}a_k)\]
    \newline \textbf{Divergence:} If the sequence of partial sums \(\{s_n\}\) is divergent (does not approach a finite limit), then we say that the series \(\displaystyle \sum^\infty_{k=1}a_k\) is \underline{divergent}.
    \newline \textbf{Absolute Convergence:} A series \(\displaystyle \sum^\infty_{n=1}a_n\) is \underline{absolutely convergent} if \(\displaystyle \sum^\infty_{n=1}|a_n|\) is convergent.
\end{definition}

\subsection{Geometric series, P-series, Alternating series, and their Convergence}
\begin{definition}
    We observe the following,
    \newline \textbf{Geometric series:} The \underline{geometric series} \(\displaystyle\sum^\infty_{n=0}x_n\) is convergent \(\iff |x| < 1\). When \(|x|<1\), we have \(\displaystyle \sum^\infty_{n=0}x_n=\frac{1}{1-x}\).
    \newline \textbf{P-series:} Let \(p \in \mathbb{R}\). For a series of the form \(\displaystyle \sum^\infty_{n=1}\frac{1}{n^p}\) called the p-series, we have
    \begin{enumerate}[label=(\roman*)]
        \item If \(p \leq 1\), then \(\displaystyle \sum^\infty_{n=1}\frac{1}{n^p}\) is divergent
        \item If \(p > 1\), then \(\displaystyle \sum^\infty_{n=1}\frac{1}{n^p}\) is convergent.
    \end{enumerate}
    \textbf{Alternating series:} Suppose \(\{b_n\}\) is a sequence in \(\mathbb{R}\) such that \(b_1 \geq b_2 \geq \dots \geq 0\) and \(b_n \to 0\). Then the alternating series of the form \(\displaystyle \sum^\infty_{n=1}(-1)^{n-1}b_n\) and \(\displaystyle \sum^\infty_{n=1}(-1)^nb_n\) are convergent.
\end{definition}

\subsection{The \(\epsilon\)-\(\delta\) definition of \(\displaystyle \lim_{x \to c}f(x)\) and its negation}
\begin{definition}
    Let \((X,d_X)\) and \((Y,d_Y)\) be metric spaces, \(E \subseteq X, \ c \in E', \ f: E \to Y, \text{ and } q \in Y.\) We write the \(\epsilon\)-\(\delta\) definition of \(\displaystyle \lim_{x \to c}f(x)=q\) as 
    \[\underline{\forall \epsilon > 0, \exists \delta > 0, \forall x \in E \text{ with } 0 < d_X(x,c) < \delta, d_Y(f(x),q) < \epsilon}\]
    That is, \[x \in \Big(B_X(c,\delta)-\{c\}\Big)\cap E \implies f(x) \in B_Y(q, \epsilon), \text{ or equivalently, }\]
    \[f\Big((B_X(c,\delta)-\{c\})\cap E\Big) \subseteq B_Y(q, \epsilon)\]
    \newline \textbf{Negation:} The \(\epsilon\)-\(\delta\) description of the \underline{negation} of \(\displaystyle \lim_{x \to c}f(x)=q\) is
    \[\underline{\exists \epsilon_0 > 0, \forall \delta > 0, \exists x \in E \text{ with } 0 < d_X(x,c)< \delta, \ d_Y(f(x), q) \geq \epsilon_0}\]
\end{definition}

\subsection{The \(\epsilon\)-\(\delta\) definition of continuity of \(f\) at \(c\) and its negation}
\begin{definition}
    Let \((X,d_X)\) and \((Y,d_Y)\) be metric spaces, \(c \in X\), and \(f:X \to Y\). We say that \(f\) is \underline{continuous at c} and write the \(\epsilon\)-\(\delta\) definition if
    \[\underline{\forall \epsilon > 0, \exists \delta = \delta(\epsilon, c) > 0, \forall x \text{ with } d_X(x,c)<\delta, \ d_Y(f(x), f(c))< \epsilon}\]
    That is, \(f\Big(B_X(c,\delta)\Big)\subseteq B_Y\Big(f(c),\epsilon\Big)\). If \(f\) is continuous at every point in \(X\), then we say that \(f\) is \underline{continuous on \(X\)}.
    \newline \textbf{Negation:} Therefore, we can write the negation and say that \(f\) is not continuous at \(c\) if and only if
    \[\underline{\exists \epsilon_0>0, \forall \delta > 0, \exists x \text{ with } d_X(x,c) < \delta, \ d_Y(f(x),f(c)) \geq \epsilon_0}\]
\end{definition}

\subsection{The \(\epsilon\)-\(\delta\) definition of uniform continuity of \(f\)}
\begin{definition}
    Let \((X,d_X)\) and \((Y,d_Y)\) be metric spaces, and let \(f:X\to Y\). We say that f is \underline{uniformly continuous on X} if
    \[\underline{\forall \epsilon>0, \exists \delta = \delta(\epsilon) > 0, \forall p, q \in X \text{ with } d_X(p,q) < \delta, \ d_Y(f(p),f(q)) < \epsilon}\]
    \textbf{Negation:} We say that the negation, \(f:X \to Y\) is \underline{not uniformly continuous on X} if and only if
    \[\underline{\exists \epsilon_0>0, \forall \delta > 0, \exists p, q \in X \text{ with } d_X(p,q) < \delta, \ d_Y(f(p),f(q)) \geq \epsilon_0}\]
\end{definition}

\section{Results}
\subsection{Chapter 3: Theorem 11}
Let \((X,d)\) be a metric space and let \(\{x_n\}\) be a sequence in \(X\).
\begin{enumerate}[label=(\roman*)]
    \item If \(\{x_n\}\) is convergent, then \(\{x_n\}\) is Cauchy.
    \item If \(\{x_n\}\) is Cauchy, then \(\{x_n\}\) is bounded.
\end{enumerate}

\subsection{Chapter 3: Theorem 14}
\textbf{(Cauchy Criterion for Series)}. Let \(\displaystyle \sum^\infty_{n=1}a_n\) be a series in \(\mathbb{R}\). Then
\[\sum^\infty_{n=1}a_n \text{ is convergent } \iff \forall \epsilon > 0, \exists N, \forall m \geq n \geq N, \Bigg|\sum^\infty _{n=1}a_k \Bigg| < \epsilon\]

\subsection{Chapter 3: Corollary 7}
Let \(\displaystyle \sum^\infty_{n=1}a_n\) be a series in \(\mathbb{R}\). Then
\[\sum^\infty_{n=1}a_n \text{ is divergent } \iff \exists \epsilon_0 > 0, \forall N, \exists m \geq n \geq N, \Big|\sum^m_{k=n}a_k\Big| \geq \epsilon_0\]
Note that if \(\displaystyle \sum^\infty_{n=1}a_n\) is convergent, then \(a_n \to 0\) since 
\[a_n = s_n - s_{n-1} \to s - s = 0\]
Therefore, we have the following test for divergence.

\subsection{Chapter 3: Theorem 15}
\textbf{(Divergence Test)}. If \(a_n \not \to 0\), then \(\displaystyle \sum^\infty_{n=1}a_n\) is divergent.

\subsection{Chapter 3: Theorem 16}
If \(\displaystyle \sum^\infty_{n=1}a_n\) is absolutely convergent, then \(\displaystyle \sum^\infty_{n=1}a_n\) is convergent.

\subsection{Chapter 3: Corollary 6}
\textbf{(Cauchy Criterion)}. Let \(\{x_n\}\) be a sequence in \(\mathbb{R}\). Then 
\[\{x_n\} \text{ is convergent } \iff \{x_n\} \text{ is Cauchy.}\]

\subsection{Chapter 3: Corollary 8}
\textbf{(Comparison Theorem)}. Let \(\displaystyle \sum^\infty_{n=1}a_n\) and \(\displaystyle \sum^\infty_{n=1}b_n\) be nonnegative series. Suppose that \(\displaystyle \exists n_0 \in \mathbb{N}\) such that \(\displaystyle 0 \leq a_n \leq b_n\) for all \(\displaystyle n \geq n_0\).
\begin{enumerate}[label=(\roman*)]
    \item If \(\displaystyle \sum^\infty_{n=1}a_n\) is divergent, then \(\displaystyle \sum^\infty_{n=1}b_n\) is divergent.
    \item If \(\displaystyle \sum^\infty_{n=1}b_n\) is convergent, then \(\displaystyle \sum^\infty_{n=1}a_n\) is convergent.
\end{enumerate}

\subsection{Chapter 4: Theorem 2}
\textbf{(Sequential Criterion for Convergence)}. Let \((X,d_X)\) and \((Y,d_Y)\) be metric spaces, \(E \subseteq X, \ c \in E', \ f:E \to Y, \text{ and } q \in Y\). Then
\[\lim_{x \to c}f(x)=q \iff \forall \text{ sequence } \{c_n\} \text{ in } E - \{c\}, \ c_n \to c \implies f(c_n) \to q.\]

\subsection{Chapter 4: Theorem 8}
Let \(X\) and \(Y\) be metric spaces with \(X\) compact and let \(f : X \to Y\) be continuous. Then \(f(X)\) is a compact set in \(Y\). Therefore, \(f(X)\) is bounded and closed in \(Y\).

\subsection{Chapter 4: Corollary 1}
We have
\[f(x) \not \to q \text{ as } x \to c \iff \exists \text{ sequence } \{c_n\} \text{ in } E - \{c\} \text{ such that } c_n \to c \text{ but } f(c_n) \not \to q.\]

\subsection{Chapter 4: Corollary 2}
If there exist sequences \(\{x_n\}\) and \(\{y_n\}\) in \(E-\{c\}\) such that 
\[x_n \to c, \ y_n \to c, \ f(x_n) \to p, \ f(y_n) \to q \text{ and } p \not = q, \text{ then } \displaystyle \lim_{x \to c}f(x) \text{ DNE.}\]

\subsection{Chapter 4: Corollary 3}
\textbf{(Sequential Criterion for Continuity)}. Let \(c \in X\). Then
\[f: X \to Y \text{ is continuous at } c \iff \Big[c_n \to c \text{ in } X \implies f(c_n) \to f(c) \text{ in } Y \Big ].\]

\subsection{Chapter 4: Corollary 7}
If \(f:X \to Y\) is continuous, then \(\forall\) compact \(E \subseteq X, \ f(E)\) is compact.


\section{Questions/Proofs}
\subsection{Assignment 6: Question \#1}
\textbf{Question:} Suppose that \(a_n > 0 \ \ (n = 1,2, \dots)\).
\begin{enumerate}[label=(\roman*)]
    \item Prove that if \(\displaystyle \sum a_n\) is convergent, then \(\displaystyle\sum \frac{\sqrt{a_n}}{n}\) is convergent.
    \begin{solution*}
        \textit{Analysis on the property we will be using:}
        \[(a-b)^2 = a^2 -2ab + b^2 \ge 0 \implies ab \le \frac{1}{2}(a^2 + b^2)\]
    \begin{proof}
        We have
        \[\frac{\sqrt{a_n}}{n} = \Big(\sqrt{a_n} \Big) \Bigg(\frac{1}{n} \Bigg) \le \frac{1}{2}\Big(a_n + \frac{1}{n^2} \Big)\]
        Since 
        \[ \sum a_n \text{ and } \sum \frac{1}{n^2} \text{ converge} \implies \displaystyle \sum \frac{1}{2} \Big(a_n + \frac{1}{n^2} \Big) \text{ converges.} \]
        \newline Therefore, since \(\displaystyle \sum \frac{\sqrt{a_n}}{n} \le \sum \frac{1}{2} \Big(a_n + \frac{1}{n^2} \Big)\), then \(\displaystyle \sum \frac{\sqrt{a_n}}{n}\) converges.
    \end{proof}
    \end{solution*}
    \item Prove that if \(\displaystyle \sum a_n\) is divergent, then \(\displaystyle \sum \frac{a_n}{1 + a_n}\) is divergent.
    \begin{solution*}
        We will use the contrapositive form: 
        \[\displaystyle \sum \frac{a_n}{1+ a_n} \text{ converges} \implies \displaystyle \sum a_n \text{ converges}\]
        \begin{proof}
            Assume \(\displaystyle \sum \frac{a_n}{1 + a_n}\) converges.
            \begin{align*}
                & \implies \lim_{n \to \infty} \frac{a_n}{1+a_n} = 0, \text{ so } \exists N, \forall n \ge N, \frac{a_N}{1+a_n} < \frac{1}{2} \\
                & \implies 2a_n < 1 + a_n \text{ or } a_n < 1 \text{ for all } n \ge N \\
                & \implies 1 + a_n < 2 \\
                & \implies \frac{1}{2} < \frac{1}{1 + a_n} \\
                & \implies \frac{1}{2}a_n < \frac{a_n}{1 + a_n}
            \end{align*}
            So, we have that \(\displaystyle \sum \frac{1}{2}a_n\) converges since \(\displaystyle \sum \frac{a_n}{1+ a_n}\) converges.
            \newline Therefore, \(\displaystyle \sum a_n\) is convergent.
        \end{proof}
    \end{solution*}
\end{enumerate}

\subsection{Assignment 6: Question \#2}
\textbf{Question:} Discuss the (absolute) convergence/divergence of the series \(\displaystyle \sum^\infty _{n=1}a_n\), where \(a_n\) is given by
\begin{multicols}{2}
\begin{enumerate}[label =(\roman*)]
    \item \(a_n = (-1)^n \Big(1 + \frac{1}{n} \Big)\)
    \item \(a_n = \sqrt{n + 1} - \sqrt{n}\)
    \item \(a_n = (-1)^n \frac{1}{\sqrt{n}}\)
    \item \(a_n = \frac{\sqrt{n+1}-\sqrt{n}}{2n}\)
    \item \(a_n = \frac{(-1)^n}{1+x^n}, (x \geq 0)\)
    \item \(a_n = \frac{n!}{1000^n}\)
    \item \(a_n = \frac{n!}{n^n}\)
    \item \(a_n = \frac{\cos n}{n^{5/4}}\)
\end{enumerate}
\end{multicols}
\begin{solution*}
    
\end{solution*}

\subsection{Assignment 7: Question \#2}
\textbf{Question:} Use the \(\epsilon\)-\(\delta\) definition to prove that
\begin{multicols}{3}
\begin{enumerate}[label =(\roman*)]
    \item \(\displaystyle \lim_{x \to 2}(x^2 - x +1) = 3\)
    \item \(\displaystyle \lim_{x \to 0}\frac{2x}{1+ x^2} = 0\)
    \item \(\displaystyle \lim_{x \to 1}\sqrt{x^2 + 2} = \sqrt{3}\)
\end{enumerate}
\end{multicols}
\begin{solution*}
    
\end{solution*}

\subsection{Assignment 7: Question \#5}
\textbf{Question:} Let \(\displaystyle f(x) = \begin{cases}
    \displaystyle x \sin\Big(\frac{1}{x}\Big) & \text{if } x \ne 0 \\
    0 & \text{if } x = 0
\end{cases}\)
Use the \(\epsilon\)-\(\delta\) definition to prove that \(f : \mathbb{R} \times \mathbb{R}\) is continuous at 0.
\begin{solution*}
    
\end{solution*}

\subsection{Assignment 7: Question \#6}
\textbf{Question:} Use the \(\epsilon\)-\(\delta\) definition to prove that \(f(x) = \sqrt{x}\) is continuous at every \(c \ge 0\).
\begin{solution*}
    To prove \(f\) is continuous at every \(c \ge 0\), let's do some analysis on \(|f(x) - f(c)|\) and relate it with \(M|x - c|\) where \(M > 0\). By doing so, we can find the proper delta to use in our formal proof.
    \newline \textit{Analysis:}
    \begin{align*}
        \displaystyle |f(x)-f(c)| & = \displaystyle | \sqrt{x} - \sqrt{c} | = \begin{cases}
            \sqrt{x} & c = 0 \\
            \frac{1}{\sqrt{x} + \sqrt{c}}|x-c| \le \frac{1}{\sqrt{c}}|x-c| & c > 0
        \end{cases}
    \end{align*}
    When \(c = 0\), we need to choose a delta such that
    \[ \sqrt{x} = \epsilon \implies \delta = \epsilon^2\]
    When \(c > 0\), we need to choose a delta such that
    \[ \frac{1}{\sqrt{x} + \sqrt{c}}|x-c| \le \frac{1}{\sqrt{c}}|x-c| \implies |x-c| < \epsilon \sqrt{c} \implies \delta = \epsilon \sqrt{c}\]
    Further expanding
    \[|\sqrt{x} - \sqrt{c}| = \frac{|x-c|}{\sqrt{x} + \sqrt{c}}\]
    \begin{proof}
        Given \(\epsilon > 0\).
        \newline \textbf{Case 1:} \(c = 0\)
        \newline Take \(\delta = \epsilon^2\). Then \(\delta > 0\). Now,
        \[ \forall x \in [0, \infty) \text{ with } |x-0|= x < \delta\]
        we have
        \[|f(x)-f(c)| = \sqrt{x} < \sqrt{\delta} = \epsilon\]
        Therefore, \(f\) is continuous at 0.
        \newline \textbf{Case 2:} \(c > 0\)
        \newline Take \(\delta = \epsilon \sqrt{c}\). Then \(\delta > 0\). Now,
        \[ \forall x \in [0, \infty) \text{ with } |x-c| < \delta\]
        we have
        \[|f(x)-f(c)| = |\sqrt{x} - \sqrt{c}| = \frac{1}{\sqrt{x} + \sqrt{c}}|x-c| \le \frac{1}{\sqrt{c}}|x-c| < \frac{1}{\sqrt{c}}\delta = \epsilon\]
        Therefore, \(f\) is continuous at c.
    \end{proof}
\end{solution*}

\subsection{Assignment 7: Question \#7}
\textbf{Question:} Use the \(\epsilon\)-\(\delta\) definition to prove the uniform continuity of the following functions on the given intervals.
\begin{enumerate}[label=(\roman*)]
    \item \(\displaystyle f(x) = \frac{1}{x^2} \text{ on } [2, \infty)\)
    \begin{solution*}
        To prove uniform continuity of \(f\), let's do some analysis on \(|f(p) - f(q)|\) and relate it with \(M|p - q|\) where \(M > 0\). By doing so, we can find the proper delta to use in our formal proof.
        \newline \textit{Analysis:} 
        \begin{align*}
            \displaystyle |f(p)-f(q)| & = \displaystyle \Big| \frac{1}{p^2} - \frac{1}{q^2} \Big| = \Big| \frac{q^2 - p^2}{p^2q^2} \Big| = \frac{(q+p)|p-q|}{p^2q^2} \\
            & = \displaystyle \Big ( \frac{1}{p^2q^2} + \frac{1}{pq^2} \Big)|p-q| \le \Big(\frac{1}{8} + \frac{1}{8} \Big)|p-q| \text{ (since } p,q \ge 2) \\
            & = \frac{1}{4}|p-q| < \frac{\delta}{4} = \epsilon \implies \delta = 4\epsilon
            \end{align*}
            Now we have found the correct delta to use in our formal proof.
            \begin{proof}
                Given \(\epsilon > 0\). Take \(\delta = 4\epsilon\). Then \( \delta > 0\). Now,
                \[\forall p, q \in [2, \infty) \text{ with } |p-q| < \delta\]
                we have
                \[|f(p)-f(q)| = \Big(\frac{1}{p^2q} + \frac{1}{pq^2} \Big) |p-q| \le \frac{1}{4}|p-q| < \frac{\delta}{4} = \epsilon\]
                Therefore, \(f\) is uniform continuous.
            \end{proof}
    \end{solution*}
    \item \(\displaystyle f(x) = \frac{x-1}{x+1} \text{ on } [0, \infty)\)
    \begin{solution*}
        To prove uniform continuity of \(f\), let's do some analysis on \(|f(p) - f(q)|\) and relate it with \(M|p - q|\) where \(M > 0\). By doing so, we can find the proper delta to use in our formal proof.
        \newline \textit{Analysis:} 
        \begin{align*}
            \displaystyle |f(p)-f(q)| & = \displaystyle \Big | \frac{p-1}{p+1} - \frac{q-1}{q+1} \Big | \\
            &= \frac{2|p-q|}{(p+1)(q+1)} \le 2|p-q| < 2 \delta = \epsilon \implies \delta = \frac{\epsilon}{2}
        \end{align*}
        Now we have found the correct delta to use in our formal proof.
        \begin{proof}
            Given \(\epsilon > 0\). Take \(\delta = \frac{\epsilon}{2}\). Then \(\delta > 0\). Now,
            \[\forall p, q \in [0, \infty) \text{ with } |p-q| < \delta\]
            we have
            \[\frac{2|p-q|}{(p+1)(q+1)} \le 2|p-q| < 2\delta = \epsilon\]
            Therefore, \(f\) is uniform continuous.
        \end{proof}
    \end{solution*}
    \item \(\displaystyle f(x) = \frac{x}{1+x^2} \text{ on } (-\infty, \infty)\)
    \begin{solution*}
        To prove uniform continuity of \(f\), let's do some analysis on \(|f(p) - f(q)|\) and relate it with \(M|p - q|\) where \(M > 0\). By doing so, we can find the proper delta to use in our formal proof.
        \newline \textit{Analysis:} 
        \begin{align*}
            \displaystyle |f(p)-f(q)| & = \displaystyle \Big| \frac{p}{1+p^2} - \frac{q}{1+q^2} \Big| = \frac{|p+pq^2-q-p^2q|}{(1+p^2)(1+q^2)} = \frac{|(p-q)+pq(q-p)|}{(1+p^2)(1+q^2)} \\
            & = \frac{|1-pq||p-q|}{1+p^2+q^2+p^2q^2} \le \frac{1 + |p||q|}{1 + p^2 + q^2 + p^2q^2}|p-q| \\
            & \le \frac{1+\frac{1}{2}(p^2+q^2)}{1+p^2+q^2}|p-q| \le |p-q| < \delta = \epsilon \implies \delta = \epsilon
        \end{align*}
        Now we have found the correct delta to use in our formal proof.
        \begin{proof}
            Given \(\epsilon > 0\). Take \(\delta = \epsilon\). Then \(\delta > 0\). Now,
            \[\forall p, q \in (-\infty, \infty) \text{ with } |p-q| < \delta \]
            we have
            \[|f(p)-f(q)| \le \frac{1 + |p||q|}{1 + p^2 + q^2 + p^2q^2}|p-q| \le |p-q| < \delta = \epsilon\]
            Therefore, \(f\) is uniform continuous.
        \end{proof}
    \end{solution*}
\end{enumerate}


\subsection{Chapter 2: the proof of Theorem 9}
\textbf{Theorem:} Let \(K\) be a compact set in metric space \(X\) and let \(F\) be a closed subset of \(K\). Then \(F\) is compact.
\begin{proof}
    Let \(\{G_\alpha \}\) be an open cover of \(F\). Then \(\{X-F\} \cup \{G_\alpha \}\) is an open cover of \(K\). Since \(K\) is compact, \(\exists \alpha_1, \dots, \alpha_n\) such that
    \[K \subseteq (X-F) \cup \Big(G_{\alpha_1} \cup \dots \cup G_{\alpha_n}\Big).\]
    Now \(F = F \cap K \subseteq \Big(F \cap (X-F)\Big) \cup \Big(G_{\alpha_1}\cup \dots \cup G_{\alpha_n}\Big)=G_{\alpha_1} \cup \dots \cup G_{\alpha_n}.\) Hence, \(\{G_\alpha\}\) has a finite subcover \(\{G_{\alpha_1}, \dots, G_{\alpha_n}\}\) of \(F\). Therefore, \(F\) is compact.
\end{proof}

\subsection{Chapter 3: the proof of Theorem 11}
\textbf{Theorem:} Let \((X,d)\) be a metric space and let \(\{x_n\}\) be a sequence in \(X\).
\begin{enumerate}[label=(\roman*)]
    \item If \(\{x_n\}\) is convergent, then \(\{x_n\}\) is Cauchy.
    \item If \(\{x_n\}\) is Cauchy, then \(\{x_n\}\) is bounded.
\end{enumerate}
\begin{proof}
    \begin{enumerate}[label=(\roman*)]
        \item Suppose \(x_n \to x.\) Then \(\forall \epsilon > 0, \exists N, \forall n \geq N,  d(x_n,x) < \frac{\epsilon}{2}\), and thus \(\forall m, n \geq N,\) we have
        \[d(x_m,x_n) \leq d(x_m,x) + d(x,x_n)=d(x_m,x)+d(x_n,x) < \epsilon.\]
        Therefore, \(\{x_n\}\) is Cauchy.
        \item Suppose \(\{x_n\}\) is Cauchy. For \(\epsilon = 1, \exists N,\ \forall m,  n \geq N,  d(x_m,x_n) < \epsilon\). Let \(x=x_N\) and \(r=1+\displaystyle \max_{1 \leq i \leq N} d(x_i,x)\). Then \(d(x_n,x) < r\) for all \(n\); that is, \(\ \{x_n\}\) is in the ball \(B(x,r)\). Therefore, \(\ \{x_n\}\) is bounded.
    \end{enumerate}
\end{proof}

\subsection{Chapter 4: the proof of Theorem 8}
\textbf{Theorem:} Let \(X\) and \(Y\) be metric spaces with \(X\) compact and let \(f : X \to Y\) be continuous. Then \(f(X)\) is a compact set in \(Y\). Therefore, \(f(X)\) is bounded and closed in \(Y\).
\begin{proof}
    Let \(\mathcal{U}\) be an open cover of \(f(X)\) in \(Y\); that is, \(\mathcal{U}\) is a family of open sets on Y such that \(f(X) \subseteq \displaystyle \bigcup_{U \in \ \mathcal{U}}U.\) Then
    \[X \subseteq f^{-1} \Big( \bigcup_{U \in \ \mathcal{U}}U \Big)=\bigcup_{U \in \ \mathcal{U}}f^{-1}(U)\]
    with each \(f^{-1}(U)\) open in \(X\) (since \(f\) is continuous). Thus \(\Big \{f^{-1}(U):U \in \mathcal{U} \Big\}\) is an open cover of \(X\). By the compactness of \(X, \exists U_1, \dots, U_n \in \mathcal{U}\) such that \(\displaystyle X= \bigcup^n_{k=1}f^{-1}(U_k)\). Hence,
    \[f(X)=f\Big(\bigcup^n_{k=1}f^{-1}(U_k)\Big)=\bigcup^n_{k=1}f(f^{-1}(U_k)) \subseteq\bigcup^n_{k=1}U_k.\]
    So, \(\mathcal{U}\) has a finite subcover of \(f(X)\). Therefore, \(f(X)\) is compact.
\end{proof}

\end{document}
